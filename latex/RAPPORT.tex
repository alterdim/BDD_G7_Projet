\documentclass[12pt]{article}

\usepackage{amsmath}
\usepackage{amsfonts}

\usepackage{graphicx}
\usepackage{xcolor}
\usepackage{hyperref}
\usepackage{siunitx}
\usepackage{polynom}
\usepackage[utf8]{inputenc}

\title{Bases de Données \\ Cinémas genevois}

\author{Rebeka Mali, Valon Halili, Ayman Chidda,\\ Loris Thomas, Louis Gérard}

\date{Printemps 2024}

\begin{document}
\maketitle

\section{Description générale}
Le projet est une base de données des cinémas genevois. Elle contient : 
\begin{itemize}
    \item les cinémas, leurs emplacments
    \item les salles des cinémas, leurs équipements spéciaux (type IMAX)
    \item les séances qui ont lieux dans ces salles, avec leurs horaires
    \item les films qui y sont diffusés, leurs genres, réalisateurs...
\end{itemize}
La base de données alimente une application dédiée aux cinéphiles genevois qui souhaitent
effectuer des recherches liés à ces données. Il leur est possible de trier les films par genre,
les cinémas par région, les salles par équipement spécifique pour trouver la séance qui leur convient le mieux.
\pagebreak
\section{Diagramme des cas d'utilisations}

\pagebreak
\section{Diagramme des classes}
\pagebreak
\section{Liste des attributs}
\paragraph*{Cinema}
\begin{center}
    \begin{tabular}{||c c c||} 
     \hline
     Attribut & Domaine & Synopsis\\ [0.5ex] 
     \hline\hline
     cinema{\_}id & int(10) & Identifiant du cinéma \\ 
     \hline
     cinema{\_}nom & varchar(32) & Nom du cinéma \\
     \hline
     salle{\_}nb & int(10) & Nombre de salles du cinéma\\
     \hline
     NPA & int(10) & Le code postal du cinéma [1ex] \\
     \hline
    \end{tabular}
\end{center}
\paragraph*{Salle}
\begin{center}
    \begin{tabular}{||c c c||} 
     \hline
     Attribut & Domaine & Synopsis\\ [0.5ex] 
     \hline\hline
     salle{\_}id & int(10) & Identifiant de la salle \\ 
     \hline
     salle{\_}nom & varchar(32) & Nom de la salle \\
     \hline
     cinema{\_}id & int(10) & Identifiant du cinéma\\
     \hline
     IMAX & bool & La salle est-elle équipée IMAX ? \\
     \hline
     ScreenX & bool & La salle est-ele équipée ScreenX ? \\
     \hline
     Capacité & int(10) & Capacité de la salle en personnes \\
     \hline
    \end{tabular}
\end{center}
\paragraph*{Séance}
\begin{center}
    \begin{tabular}{||c c c||} 
     \hline
     Attribut & Domaine & Synopsis\\ [0.5ex] 
     \hline\hline
     séance{\_}id & int(10) & Identifiant de la séance \\ 
     \hline
     cinema{\_}id & int(10) & Identifiant du cinéma \\
     \hline
     salle{\_}id & int(10) & Identifiant de la salle\\
     \hline
     film{\_}id & int(10) & Identifiant du film \\
     \hline
     date{\_}séance & date & Date de la séance \\
     \hline
     heure{\_}séance & heure & Heure de la séance \\
     \hline
    \end{tabular}
\end{center}
\paragraph*{Film}
\begin{center}
    \begin{tabular}{||c c c||} 
     \hline
     Attribut & Domaine & Synopsis\\ [0.5ex] 
     \hline\hline
     film{\_}id & int(10) & Identifiant du film \\ 
     \hline
     titre & varchar(32) & Titre du film \\
     \hline
     duree & time & Durée du film \\
     \hline
     date{\_}sortie & date & Date de sortie du film \\
     \hline
     pays & varchar(32) & Pays du film \\
     \hline
     réalisteur & varchar32 & Réalisateur du film \\
     \hline
    \end{tabular}
\end{center}
\paragraph*{Genre}
\begin{center}
    \begin{tabular}{||c c||} 
     \hline
     Attribut & Domaine & Synopsis\\ [0.5ex] 
     \hline\hline
     film{\_}id & int(10) & Identifiant du film \\ 
     \hline
     titre & varchar(32) & Titre du film \\
     \hline
     duree & time & Durée du film \\
     \hline
     date{\_}sortie & date & Date de sortie du film \\
     \hline
     pays & varchar(32) & Pays du film \\
     \hline
     réalisteur & varchar32 & Réalisateur du film \\
     \hline
    \end{tabular}
\end{center}
\pagebreak
\section{Schémas des relations}
\pagebreak
\section{Description des relations et justifications 3FN}
Notre base de données comporte 6 relations en total. Nous allons les décrire et justifier une par une: 
\subsection*{Lieu}
Cette relation comporte 6 attributs avec la clef "lieu{\_}id". Elle fournit à l'utilisateur les informations
relatives à l'accès au cinéma : accéssibilité pour personnes à mobilité réduite, adresse, horaires ...
\subsection*{Salle}
Cette relation dispose de 9 attributs avec la clef "salle{\_}id".
Elle est liée à la relation 'lieu' grâce à la clef étrangère 'cinema{\_}id'. Cette clef permet de situer chaque salle
dans un cinéma précis. Cette relation a plusieurs usages : elle offre des informations sur la capacité de la salle,
la technologie utilisée (IMAX, ScreenX) mais aussi sur le prix des billets.
\subsection*{Séance}
Cette relation comporte 6 attributs avec comme clef "seance{\_}id". Il s'agit d'une relation qui fournit des informations
indispensables concernant la projection d'un film : elle indique le film et la salle (à travers les clefs étrangères "salle{\_}id"
et "film{\_}id") et d'ailleurs elle affiche la date et l'heure de la séance, ainsi que la langue dans laquelle le film est projeté.
\subsection*{Film}
Cette relation contient 9 attributs avec la clef "film{\_}id". Elle fournit à l'utilisateur les films diffusés.
A part le titre du film, les informations notoires sont la durée, la date de sortie, le pays de production, l'âge minimal
et le réalisateur ainsi qu'un ou deux acteurs. Toutes ces informations sont présentes pour permettre à l'utilisateur de filtrer selon des critères précis.
\subsection*{Genre}
La relation genre est simple et ne comporte que deux attributs donc une est la clef "genre{\_}id". 
Cette relation existe car certains films peuvent avoir plusieurs genres et d'autres un seul ou aucun. Pour éviter d'avoir 
un nombre élevé d'attributs dans Film (Genre1 -> Genre5) dont la plupart seraient NULL, nous avons choisi de créer une 
relation entière avec des "id" de genre et des strings associés (pour éviter des duplicatats lors de l'utilisation de la BDD)
\subsection*{FilmGenre}
La relation film{\_}genre unit les relations Film et Genre pour attribuer un nombre indéterminé lors de la création de la BDD de genres à chaque film.
\pagebreak
\section{Requêtes SQL}
\pagebreak
\section{Conclusions}
\end{document} 