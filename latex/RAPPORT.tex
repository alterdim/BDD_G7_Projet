\documentclass[12pt]{article}

\usepackage{amsmath}
\usepackage{amsfonts}
\usepackage{hyperref}
\usepackage{graphicx}
\usepackage{xcolor}
\usepackage{hyperref}
\usepackage{siunitx}
\usepackage{polynom}
\usepackage[utf8]{inputenc}
\usepackage[T1]{fontenc}

\hypersetup{
    colorlinks=true,
    linkcolor=blue,
    filecolor=magenta,      
    urlcolor=blue,
    pdftitle={Overleaf Example},
    pdfpagemode=FullScreen,
    }

\title{Bases de Données \\ Cinémas genevois}

\author{Rebeka Mali, Valon Halili, Ayman Chidda,\\ Loris Thomas, Louis Gérard}

\date{Printemps 2024}

\begin{document}
\maketitle
\pagebreak
\tableofcontents
\pagebreak
\section{Description générale}
Le projet est une base de données des cinémas genevois. Elle contient : 
\begin{itemize}
    \item les cinémas, leurs emplacements
    \item les salles des cinémas, leurs équipements spéciaux (type IMAX)
    \item les séances qui ont lieux dans ces salles, avec leurs horaires
    \item les films qui y sont diffusés, leurs genres, réalisateurs...
\end{itemize}
La base de données alimente une application dédiée aux cinéphiles genevois qui souhaitent
effectuer des recherches liés à ces données. Il leur est possible de trier les films par genre,
les cinémas par emplacement, les salles par équipement spécifique pour trouver la séance qui leur convient le mieux. \\
La base de données que nous avons créée utilise la technologie \href{https://www.mysql.com/}{MySQL}. Pour donner une meilleure
idée d'une utilisation possible de cette base, nous rendons également une maquette d'application Web qui utilise le framework
\href{https://streamlit.io/}{Streamlit}. Elle est disponible à \href{https://bddg7unige.streamlit.app/}{cette adresse} et son code source
est également inclus dans le rendu du projet (voir README.md pour la build). La base de données est déployée
sur le service gratuit Clever Cloud pour pouvoir être utilisée par l'application Streamlit.  \\

Les données de la base sont un échantillon représentatif des sorties de films de ces derniers mois. Dans le cas 
d'un déploiement pratique de cette application, il faudrait qu'elle contienne idéalement autant de films que possible mais
ceux qui sont présents actuellement suffisent amplement à présenter les fonctionnalités de la base.
\pagebreak
\section{Description des relations}
Notre base de données comporte 6 relations en total. Nous allons les décrire et justifier une par une: 
\subsection{Lieu}
Cette relation comporte 6 attributs avec la clef "lieu{\_}id". Elle fournit à l'utilisateur les informations
relatives à l'accès au cinéma : accéssibilité pour personnes à mobilité réduite, adresse, horaires ...
\subsection{Salle}
Cette relation dispose de 9 attributs avec la clef "salle{\_}id".
Elle est liée à la relation 'lieu' grâce à la clef étrangère 'cinema{\_}id'. Cette clef permet de situer chaque salle
dans un cinéma précis. Cette relation a plusieurs usages : elle offre des informations sur la capacité de la salle,
la technologie utilisée (IMAX, ScreenX) mais aussi sur le prix des billets.
\subsection{Séance}
Cette relation comporte 6 attributs avec comme clef "seance{\_}id". Il s'agit d'une relation qui fournit des informations
indispensables concernant la projection d'un film : elle indique le film et la salle (à travers les clefs étrangères "salle{\_}id"
et "film{\_}id") et d'ailleurs elle affiche la date et l'heure de la séance, ainsi que la langue dans laquelle le film est projeté.
\subsection{Film}
Cette relation contient 9 attributs avec la clef "film{\_}id". Elle fournit à l'utilisateur les films diffusés.
A part le titre du film, les informations notoires sont la durée, la date de sortie, le pays de production, l'âge minimal
et le réalisateur ainsi qu'un ou deux acteurs. Toutes ces informations sont présentes pour permettre à l'utilisateur de filtrer selon des critères précis.
\subsection{Genre}
La relation genre est simple et ne comporte que deux attributs donc une est la clef "genre{\_}id". 
Cette relation existe car certains films peuvent avoir plusieurs genres et d'autres un seul ou aucun. Pour éviter d'avoir 
un nombre élevé d'attributs dans Film (Genre1 à Genre5) dont la plupart seraient NULL, nous avons choisi de créer une 
relation entière avec des "id" de genre et des strings associés (pour la validation de données et éviter
d'avoir 13 genres "comédie" avec des orthographes différentes).
\subsection{Film{\_}genre}
La relation film{\_}genre unit les relations Film et Genre pour attribuer un nombre indéterminé lors de la création de la BDD de genres à chaque film.
\section{Diagramme des cas d'utilisations}
\includegraphics*[scale=0.48]{images/Use-Case.drawio.png}


Dans les cas d’utilisation, nous avons choisi deux acteurs principaux : 
le client et l’administrateur.  
Pour ce qui concerne le client, 
nous avons envisagé comme cas d’utilisation : 
consulter le prix d’un billet, chercher l’horaire d’un film, 
chercher un genre particulier (par exemple s’il veut lancer une 
recherche de tous les films d’horreur qui sont disponibles). 
D'autres cas d’utilisation sont la recherche d’un cinéma et la recherche 
d’un film : dans le premier cas, le client pourrait choisir de 
chercher un cinéma qui a un accès handicap (le fait d’être optionnel est 
marqué par la relation <<extend>>) ; dans le deuxième cas, lorsqu’il 
cherche un film, le client a l’option de chercher aussi le genre d’un film 
s’il désire. Néanmoins, afin de chercher un film, il doit d’abord consulter 
la liste des films.  \\
 
Pour ce qui concerne l’administrateur du système, il y a plusieurs cas d’utilisation 
: modifier le prix d’un billet, ajouter une nouvelle séance,
 effacer une séance, changer l’adresse d’un cinéma 
 (si par exemple le cinéma est déplacé à une nouvelle adresse), 
 modifier le numéro de salles disponibles (ceci peut être dû à plusieurs raisons comme 
 l’ajout d’une nouvelle salle, son entretien ou dans le cas d’une panne), modifier 
 l’âge minimal pour un film (par exemple suite à des plaintes de la part des clients), 
 et modifier un acteur (cela pourrait arriver dans le cas où un acteur décide de 
 changer son nom d'artiste, s’il/elle se marie, etc.). 
 Certains cas d’utilisation méritent un peu plus de précisions. 
 Par exemple, modifier la liste des films engendre un choix de la part 
 de l’administrateur : il peut choisir entre effacer un film ou en ajouter. Cependant, il ne peut modifier la liste sans avoir consulté la liste des films, d’où la relation <<include>>. Lorsqu’on veut modifier une séance, le programme doit nous demander si on veut modifier l’heure ou la date. Ceci est très pratique : nous pourrions imaginer que lorsqu’un nouveau film sort, nous aimerions le projeter chaque mardi et jeudi pendant 4 mois. Cependant, nous pourrions avoir oublié qu’un mardi est un jour férié et donc soit nous avons des horaires réduits soit le cinéma est fermé. Cette option nous permet de changer facilement l’horaire et/ou la date. 

\section{Diagramme des classes}
    \includegraphics*[scale=0.7]{images/diagramme_classes.png}
Les détails de chaque classe sont inclus dans la section suivante.

\section{Liste des attributs}
Les clefs de chaque classe sont en {\color{blue} bleu}. On utilise le domaine "bool" qui n'existe pas vraiment
sur certaines technologies SQL, qui sera alors remplacé par un tinyint(1).
\label{sec:4}
\paragraph*{Lieu}
\begin{center}
    \begin{tabular}{||c c c||} 
     \hline
     Attribut & Domaine & Synopsis\\ [0.5ex] 
     \hline\hline
     {\color{blue} lieu{\_}id} & big int & Identifiant du cinéma \\ 
     \hline
     nom & varchar(50) & Nom du cinéma \\
    \hline
     adresse & varchar(50) & Adresse du cinéma \\
     \hline
     salle{\_}nb & int & Nombre de salles du cinéma\\
     \hline
     NPA & int & Le code postal du cinéma [1ex] \\
     \hline
     acces{\_}handicap & bool & Le cinéma a-t-il un accès mobilité réduite ? \\
     \hline
    \end{tabular}
\end{center}
\paragraph*{Salle}
\begin{center}
    \begin{tabular}{||c c c||} 
     \hline
     Attribut & Domaine & Synopsis\\ [0.5ex] 
     \hline\hline
     {\color{blue} salle{\_}id} & int & Identifiant de la salle \\ 
     \hline
     salle{\_}nom & varchar(50) & Nom de la salle \\
     \hline
     cinema{\_}id & int & Identifiant du cinéma\\
     \hline
     IMAX & bool & La salle est-elle équipée IMAX ? \\
     \hline
     screenX & bool & La salle est-ele équipée ScreenX ? \\
     \hline
     capacité & int & Capacité de la salle en personnes \\
     \hline
     prix{\_}plein & int & Prix tarif régulier \\
     \hline
     prix{\_}reduit & int & Prix tarif étudiant, senior... \\
     \hline
     prix{\_}enfant & int & Prix tarif enfant \\
     \hline
    \end{tabular}
\end{center}
\pagebreak
\paragraph*{Séance}
\begin{center}
    \begin{tabular}{||c c c||} 
     \hline
     Attribut & Domaine & Synopsis\\ [0.5ex] 
     \hline\hline
     {\color{blue} seance{\_}id} & int & Identifiant de la séance \\ 
     \hline
     cinema{\_}id & int & Identifiant du cinéma \\
     \hline
     salle{\_}id & int & Identifiant de la salle\\
     \hline
     film{\_}id & int & Identifiant du film \\
     \hline
     date{\_}seance & date & Date de la séance \\
     \hline
     heure{\_}seance & heure & Heure de la séance \\
     \hline
     vo & bool & La séance est-elle en Version Originale ? \\
     \hline
    \end{tabular}
\end{center}
\paragraph*{Film}
\begin{center}
    \begin{tabular}{||c c c||} 
     \hline
     Attribut & Domaine & Synopsis\\ [0.5ex] 
     \hline\hline
     {\color{blue} film{\_}id} & int & Identifiant du film \\ 
     \hline
     titre & varchar(50) & Titre du film \\
     \hline
     duree & time & Durée du film \\
     \hline
     date{\_}sortie & date & Date de sortie du film \\
     \hline
     pays & varchar(50) & Pays du film \\
     \hline
     acteur1 & varchar(50) & Acteur principal du film \\
     \hline
     acteur2 & varchar(50) & Acteur secondaire du film \\
     \hline
     réalisteur & varchar(50) & Réalisateur du film \\
     \hline
     age{\_}minmimal & int & Age minimal pour voir le film \\
     \hline
    \end{tabular}
\end{center}
\paragraph*{Genre}
\begin{center}
    \begin{tabular}{||c c c||} 
     \hline
     Attribut & Domaine & Synopsis\\ [0.5ex] 
     \hline\hline
     {\color{blue} genre{\_}id} & int & Identifiant du genre \\ 
     \hline
     nom{\_}genre & varchar(50) & Nom textuel du genre (utile pour streamlit) \\
     \hline
    \end{tabular}
\end{center}
\paragraph*{Film{\_}Genre}
\begin{center}
    \begin{tabular}{||c c c||} 
     \hline
     Attribut & Domaine & Synopsis\\ [0.5ex] 
     \hline\hline
     {\color{blue}film\_id} & int & Identifiant du film \\ 
     \hline
     {\color{blue}genre{\_}id} & int & Identifiant du genre \\
     \hline
    \end{tabular}
\end{center}
\pagebreak
\section{Schémas des relations}
\includegraphics*[scale=0.4]{images/schemarelations.png}
\paragraph*{{\color{cyan}lieu{\_}id}}
\begin{itemize}
    \item \textbf{{\color{cyan}lieu{\_}id}} dans la table Lieu est appelé clé primaire.
    \item \textbf{{\color{cyan}lieu{\_}id}} dans la table Salle est appelé clé étrangère.
\end{itemize}
\paragraph*{{\color{magenta}salle{\_}id}}
\begin{itemize}
    \item \textbf{{\color{magenta}salle{\_}id}} dans la table Salle est appelé clé primaire.
    \item \textbf{{\color{magenta}salle{\_}id}} dans la table Seance est appelé clé étrangère.
\end{itemize}
\pagebreak
\paragraph*{{\color{orange}film{\_}id}}
\begin{itemize}
    \item \textbf{{\color{orange}film{\_}id}} dans la table Film est appelé clé primaire.
    \item \textbf{{\color{orange}film{\_}id}} dans la table Film{\_}Genre est appelé clé primaire.
    \item \textbf{{\color{orange}film{\_}id}} dans la table Seance est appelé clé étrangère.
\end{itemize}
\paragraph*{{\color{violet}genre{\_}id}}
\begin{itemize}
    \item \textbf{{\color{violet}genre{\_}id}} dans la table Genre est appelé clé primaire.
    \item \textbf{{\color{violet}genre{\_}id}} dans la table Film{\_}Genre est appelé clé étrangère.
\end{itemize}
\pagebreak
\section{Justifications 3FN}
Afin de démontrer que les relations sont en 3FN, 
nous avons besoin de démontrer qu'elles sont d'abord en 1FN, 2FN, et ensuite en 3FN.
\subsection{Lieu}
\textbf{Lieu(lieu{\_}id // nom, adresse, salles{\_}nb, npa, acces{\_}handicap)}
\begin{itemize}
    \item 1FN : La relation est en première forme normale car elle possède une clef(lieu{\_}id) 
    et ses attributs sont atomiques (‘adresse’ dans ce cas est un ‘string’). 
    Nous avons choisi d’avoir comme cleflieu{\_}id car le nom d’un cinéma pourrait 
    changer au cours des années, mais le cinéma garderait la même adresse, les numéros de salles, etc. 
    Donc c’est plus simple d’assigner un id qui reste inchangé. D’ailleurs, cela facilite aussi les requêtes SQL.
    \item 2FN : Vu que la relation est en 1FN et que nous n'avons qu'un seul attribut clef, la relation est aussi
    en deuxième forme normale.
    \item 3FN : La relation est en 2FN et tous les attributs dépendent uniquement de la clef. (Le NPA ne dépend pas de l'adresse
    puisqu'il existe des doublons d'adresses justement différenciés par le NPA en Suisse).
\end{itemize}
\subsection{Salle}
\textbf{Salle(salle{\_}id // nom{\_}salle, lieu{\_}id, imax, screenx, capacite, prix{\_}plein, prix{\_}reduit, prix{\_}enfant)} \\
\begin{itemize}
    \item 1FN : La relation est en première forme normale car elle 
    possède une clef(salle{\_}id) et ses attributs sont atomiques. 
    Nous avons choisi d’avoir comme clefsalle{\_}id car deux cinémas 
    pourraient avoir le même nom pour une salle : par exemple, nous pourrions avoir la 
    salle "A" au Nord-Sud et l’autre à Les Scala. Avec un id nous sommes sûrs d’avoir la salle d’un cinéma précis. 
    \item 2FN : Vu que la relation est en 1FN et que nous n'avons qu'un seul attribut clef, la relation est aussi
    en deuxième forme normale.
    \item 3FN : la relation est en 2FN et tous les attributs dépendent de la clef. 
    En effet, toutes les technologies (imax, screenx) dépendent de salle{\_}id et 
    non pas du lieu{\_}id car un même cinéma peut avoir diverses salles équipées différemment. 
    Et, comme mentionné avant, les attributs ne peuvent pas dépendre de "nom {\_}salle", 
    car si nous avons deux mêmes noms de salle dans deux cinémas différents, 
    on risque d’avoir des valeurs d’attributs qui ne correspondent pas à la réalité.
\end{itemize}
\subsection{Séance}
\textbf{Seance(seance{\_}id//  salle{\_}id, film{\_}id, date{\_}seance, heure{\_}seance, vo)}
\begin{itemize}
    \item 1FN : La relation est en première forme normale car elle 
    possède une clef(seance{\_}id) et ses attributs sont atomiques. 
    Nous aurions pu utiliser comme clef(salle{\_}id film{\_}id date{\_}seance heure{\_}seance//) 
    mais, si nous voulions utiliser la clefde la relation comme clefétrangère dans une autre relation, 
    cela nous obligerait de répéter toutes les valeurs des attributs au lieu 
    d’une simple clefcomme seance{\_}id. Le choix d’une clefsimplifie aussi les requêtes SQL. 
    \item 2FN : Vu que la relation est en 1FN et que nous n'avons qu'un seul attribut clef, la relation est aussi
    en deuxième forme normale.
    \item 3FN : la relation est en 2FN et tous les attributs dépendent de la clef. Un même film{\_}id peut 
    être transmis dans différentes salles à différentes dates et en 
    différentes langues. Ce ne sera que le seance{\_}id qui pourra indiquer la juste combinaison. 
\end{itemize}
\pagebreak
\subsection{Film}
\textbf{Film(film{\_}id // titre, duree, date{\_}sortie, pays, acteur1, acteur2, realisateur, age{\_}minimal)}
\begin{itemize}
    \item 1FN : La relation est en première forme normale car elle possède une clef(film{\_}id) et 
    ses attributs sont atomiques (réalisateur dans ce cas est un ‘string’). Nous aurions pu utiliser 
    comme clef(titre date{\_}sortie//) mais, si nous voulions utiliser la clefde la relation Film comme 
    clefétrangère dans une autre relation, cela nous obligerait de répéter les valeurs de ces attributs dans 
    une autre relation au lieu d’une clefsimple comme film{\_}id. Ce choix nous 
    facilite les requêtes SQL et évite aussi des fautes des frappe qui pourraient arriver quand on saisit le nom du film. 
    \item 2FN : Vu que la relation est en 1FN et que nous n'avons qu'un seul attribut clef, la relation est aussi
    en deuxième forme normale.
    \item 3FN : La relation est en 2FN et tous les attributs dépendent de la clef. 
    En effet, l’attribut ‘date{\_}sortie’ ne peut dépendre de l’attribut ‘titre’ 
    car nous pourrions avoir deux films avec le même titre sortis à deux dates différentes. 
    Cela pourrait arriver si le cinéma décide de projeter l’ancien film avec le nouveau 
    qui vient de sortir avec le même titre : 
    par exemple, le film ‘Rebecca’ sorti en 1940 et en 2020.
\end{itemize}
\subsection{Genre}
\textbf{Genre(genre{\_}id//nom{\_}genre)}
\begin{itemize}
    \item 1FN : La relation est en première forme normale car elle possède une clef(genre{\_}id) et ses attributs sont atomiques (‘nom{\_}genre’ est un string). 
    \item 2FN : Vu que la relation est en 1FN et que nous n'avons qu'un seul attribut clef, la relation est aussi
    en deuxième forme normale.
    \item 3FN : La relation est en 2FN et le seul attribut de la relation dépend de la clef.
\end{itemize}
\pagebreak
\subsection{Film{\_}genre}
\textbf{Film{\_}genre(FilmID GenreID //)}
\begin{itemize}
    \item 1FN : La relation est en première forme normale car elle possède une clef(‘Film ID GenreID’); nous n’avons pas dans ce cas d’autres attributs. 
    \item 2FN : Vu que la relation est en 1FN et que nous n'avons qu'un seul attribut clef, la relation est aussi
    en deuxième forme normale.
    \item 3FN : La relation est en 2FN et le seul attribut de la relation dépend de la clef.
\end{itemize}
\pagebreak
\section{Requêtes SQL}
Voici des exemples de requêtes SQL plus ou moins générales qui donnent une idée de ce dont la base de données est capable.
Ces requêtes peuvent être effectuées automatiquement sur le site de l'application à \href{https://bddg7unige.streamlit.app/SQL}{cette adresse}.
\paragraph*{1. Les cinémas qui diffusent les films italiens pour lesquels l'âge minimal est de 12 ans}
\begin{verbatim}
SELECT DISTINCT l.nom  
FROM lieu l 
JOIN salle s ON s.lieu_id = l.lieu_id  
JOIN seance s2 ON s2.salle_id = s.salle_id  
JOIN film f ON f.film_id = s2.film_id  
WHERE f.pays LIKE '%Italie%' AND f.age_minimal = 12 
\end{verbatim}
\paragraph*{2. Les réalisateurs qui ont joué dans leur propre film, ainsi que le nom du dit film}
\begin{verbatim}
SELECT realisateur, titre 
FROM film f  
WHERE FIND_IN_SET(f.acteur1, f.realisateur) > 0 
OR FIND_IN_SET(f.acteur2, f.realisateur) > 0 
\end{verbatim}
On utilise FIND IN SET car les films ont parfois plusieurs réalisateurs, 
c'est donc une liste qui se trouve dans la colonne réalisateur.
\paragraph*{3. Les films et leurs genres avec une date de sortie en janvier 2024, qui sont encore diffusés entre le 13 et 19 mai}
\begin{verbatim}
SELECT f.titre, GROUP_CONCAT(DISTINCT g.nom_genre SEPARATOR ', ') AS genres 
FROM seance s  
JOIN film f ON f.film_id = s.film_id  
LEFT JOIN film_genre fg ON fg.film_id = f.film_id  
LEFT JOIN genre g ON g.genre_id = fg.genre_id  
WHERE f.date_sortie BETWEEN '2024-01-01' AND '2024-01-31' 
AND s.date_seance BETWEEN '2024-05-13' AND '2024-05-19'  
GROUP BY f.titre 
\end{verbatim}
\paragraph*{4. Les films d'animation sortis en l'an 2000}
\begin{verbatim}
SELECT f.titre
FROM film f  
JOIN film_genre fg ON fg.film_id = f.film_id  
JOIN genre g ON g.genre_id = fg.genre_id  
WHERE g.nom_genre = 'Dessin animé' AND f.date_sortie BETWEEN 
'2000-01-01' AND '2000-12-31' 
\end{verbatim}
\paragraph*{5. Le nombre moyen de séances d'un film sorti le 8 mai 2024 entre sa sortie et le 14 mai, par cinéma}
\begin{verbatim}
WITH seance_nb AS( 
SELECT f.titre, s2.lieu_id, COUNT(*) AS Nb 
FROM seance s  
JOIN film f ON f.film_id = s.film_id  
JOIN salle s2 ON s2.salle_id = s.salle_id  
WHERE f.date_sortie = '2024-05-08' AND s.date_seance BETWEEN 
'2024-05-08' AND '2024-05-14' 
GROUP BY s.film_id, s2.lieu_id  
) 
SELECT l.nom, AVG(Nb) AS nombre_moyen_de_seances_pour_un_nouveau_film 
FROM seance_nb 
JOIN lieu l ON l.lieu_id = seance_nb.lieu_id 
GROUP BY seance_nb.lieu_id 
\end{verbatim}
\paragraph*{6. Tous les films suisses sortis au 21ème siècle}
\begin{verbatim}
SELECT titre 
FROM film f  
WHERE date_sortie >= 2001-01-01 AND pays LIKE '%Suisse%' 
\end{verbatim}
\pagebreak
\paragraph*{7. Les acteurs jouant dans plusieurs films. Trier par nombre de films joués.}
\begin{verbatim}
SELECT acteur, COUNT(*) as nb_film  
FROM (  
SELECT f.acteur1 AS acteur  
FROM film f  
UNION ALL  
SELECT f.acteur2 AS acteur  
FROM film f  
) AS acteurs  
WHERE acteur IS NOT NULL  
GROUP BY acteur  
HAVING nb_film > 1; 
\end{verbatim}
\paragraph*{8. Les séances (Titre, heure, lieu, salle et prix) du 17 mai à partir de 17h en VO avec Drame comme l'un des genres du film}
\begin{verbatim}
SELECT f.titre, s.heure_seance, l.nom, s2.nom_salle, s2.prix_plein, 
s2.prix_reduit, s2.prix_enfant  
FROM seance s  
JOIN film f ON f.film_id = s.film_id  
JOIN film_genre fg ON fg.film_id = f.film_id  
JOIN genre g ON g.genre_id = fg.genre_id  
JOIN salle s2 ON s2.salle_id = s.salle_id  
JOIN lieu l ON l.lieu_id = s2.lieu_id  
WHERE g.nom_genre ='Drame' AND s.date_seance = '2024-05-17' 
AND s.heure_seance >= '17:00' AND s.vo = true 
\end{verbatim}
\pagebreak
\paragraph*{9. Le nombre moyen de séances par jour et par cinéma pour la semaine du 13 au 19 mai}
\begin{verbatim}
WITH nb_jour AS ( 
SELECT l.nom, s.date_seance, COUNT(*) AS nb 
FROM seance s  
JOIN salle s2 ON s2.salle_id = s.salle_id 
JOIN lieu l ON l.lieu_id = s2.lieu_id 
WHERE s.date_seance BETWEEN '2024-05-13' AND '2024-05-19' 
GROUP BY l.nom, s.date_seance  
) 
SELECT nom, AVG(nb) AS moyenne_par_jour 
FROM nb_jour 
GROUP BY nb_jour.nom
\end{verbatim}
\paragraph*{10. Les 5 salles ayant la plus grande capacité}
\begin{verbatim}
SELECT l.nom, s.nom_salle, s.capacite  
FROM salle s  
JOIN lieu l ON l.lieu_id = s.lieu_id  
ORDER BY s.capacite DESC 
LIMIT 5; 
\end{verbatim}
\paragraph*{11. Le nombre de films sortis en 2024 par genre}
\begin{verbatim}
SELECT g.nom_genre, COUNT(*) AS nb 
FROM film_genre fg  
JOIN genre g ON g.genre_id = fg.genre_id  
JOIN film f ON f.film_id = fg.film_id  
WHERE f.date_sortie BETWEEN '2024-01-01' AND '2024-12-31' 
GROUP BY g.nom_genre 
ORDER BY nb ASC 
\end{verbatim}
\pagebreak
\section{Conclusions}
\subsection{Commentaires de fin}
La base de données (et la maquette d'application associée) proposent un 
modèle évolutif simple mais efficace qui peut accompagner les évolutions 
de l'offre cinématographique genevoise
Elle est relativement simple à maintenir grace à des règles d'intégrité assez strictes mais suffisamment souple pour
permettre à l'application associée d'obtenir facilement les données qui comptent vraiment pour les utilisateurs (les séances et le
tri de films par genre).
\subsection{Limites}
Il aurait été pertinent de créer des relations Acteur, Realisateur, Pays afin d'effectuer le même
processus de validation des données que sur Film{\_}genre et s'assurer qu'il n'existe aucun doublon d'un acteur, par exemple. 
Nous avons cependant jugé que trier les films par genre était le critère le plus déterminant et pour garder la base de données
à une taille adéquate pour le projet, nous avons décidé de n'opérer cette validation que pour les genres. Pour une application
complète qui veut offrir une recherche par réalisateur, acteur ou pays en s'assurant une correcte validation des données même pour un très
grand nombre de films, il aurait fallu créer ces relations (et par conséquent une table FilmActeur pour inclure un nombre
indéterminé d'acteurs par film)
\end{document} 