\documentclass[12pt]{article}

\usepackage{amsmath}
\usepackage{amsfonts}

\usepackage{graphicx}
\usepackage{xcolor}
\usepackage{hyperref}
\usepackage{siunitx}
\usepackage{polynom}
\usepackage[utf8]{inputenc}

\title{Bases de Données \\ Cinémas genevois}

\author{Rebeka Mali, Valon Halili, Ayman Chidda,\\ Loris Thomas, Louis Gérard}

\date{Printemps 2024}

\begin{document}
\maketitle

\section{Description générale}
Le projet est une base de données des cinémas genevois. Elle contient : 
\begin{itemize}
    \item les cinémas, leurs emplacments
    \item les salles des cinémas, leurs équipements spéciaux (type IMAX)
    \item les séances qui ont lieux dans ces salles, avec leurs horaires
    \item les films qui y sont diffusés, leurs genres, réalisateurs...
\end{itemize}
La base de données alimente une application dédiée aux cinéphiles genevois qui souhaitent
effectuer des recherches liés à ces données. Il leur est possible de trier les films par genre,
les cinémas par région, les salles par équipement spécifique pour trouver la séance qui leur convient le mieux.
\pagebreak
\section{Diagramme des cas d'utilisations}
\pagebreak
\section{Diagramme des classes}
\pagebreak
\section{Liste des attributs}
\pagebreak
\section{Schémas des relations}
\pagebreak
\section{Justifications 3FN}
\pagebreak
\section{Requêtes SQL}
\pagebreak
\section{Conclusions}
\end{document} 